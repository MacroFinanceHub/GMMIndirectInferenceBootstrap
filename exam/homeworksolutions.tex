
\documentclass{article}
%%%%%%%%%%%%%%%%%%%%%%%%%%%%%%%%%%%%%%%%%%%%%%%%%%%%%%%%%%%%%%%%%%%%%%%%%%%%%%%%%%%%%%%%%%%%%%%%%%%%%%%%%%%%%%%%%%%%%%%%%%%%%%%%%%%%%%%%%%%%%%%%%%%%%%%%%%%%%%%%%%%%%%%%%%%%%%%%%%%%%%%%%%%%%%%%%%%%%%%%%%%%%%%%%%%%%%%%%%%%%%%%%%%%%%%%%%%%%%%%%%%%%%%%%%%%
\usepackage{amssymb}
\usepackage{amsmath}
\usepackage{amsfonts}
\usepackage[german]{babel}
\usepackage[a4paper]{geometry}

\setcounter{MaxMatrixCols}{10}
%TCIDATA{OutputFilter=LATEX.DLL}
%TCIDATA{Version=5.50.0.2953}
%TCIDATA{<META NAME="SaveForMode" CONTENT="1">}
%TCIDATA{BibliographyScheme=Manual}
%TCIDATA{Created=Sun Apr 19 19:12:19 1998}
%TCIDATA{LastRevised=Wednesday, February 01, 2012 16:11:21}
%TCIDATA{<META NAME="GraphicsSave" CONTENT="32">}
%TCIDATA{<META NAME="DocumentShell" CONTENT="Other Documents\Assorted - Basic Assignment">}
%TCIDATA{Language=American English}
%TCIDATA{CSTFile=LaTeX article.cst}
%TCIDATA{ComputeDefs=
%$y$
%}


\input{tcilatex}
\parindent0mm
\parskip1.5ex plus0.5ex minus0.5ex

\begin{document}

\title{Statistical Inference}
\author{Exam -- Solutions}
\date{Winter 2011/2012}
\maketitle

\section{Macroeconomic desasters}

Barro and Jin, \textquotedblleft On the Size Distribution of Macroeconomic
Desasters\textquotedblright , \emph{Econometrica}, 79 (2011) 1567-1589,
assume that extreme macroeconomic shocks follow a single power law or a
double power law distribution. The password protected pdf of the article is
downloadable from the course site. Let $0<b\leq 1$ denote the fraction of
output contraction. Barro and Jin consider the transformed disaster size%
\begin{equation}
z=\frac{1}{1-b}.  \label{e1}
\end{equation}%
The density of the double power law is%
\begin{equation*}
f_{z}(x)=\left\{ 
\begin{array}{ll}
0 & \text{if }x<z_{0} \\ 
Bx^{-\beta -1} & \text{if }z_{0}\leq x<\delta  \\ 
Ax^{-\alpha -1} & \text{if }\delta <x%
\end{array}%
\right. 
\end{equation*}%
where $\alpha ,\beta ,A,B>0$ are parameters, $z_{0}$ is a known threshold,
and $\delta >z_{0}$ is a parameter.

\begin{enumerate}
\item Show that the conditions%
\begin{eqnarray*}
\int_{z_{0}}^{\infty }f_{z}(x)dx &=&1 \\
\lim_{\varepsilon \rightarrow 0}f_{z}(\delta -\varepsilon ) &=&f_{z}(\delta )
\end{eqnarray*}%
with $\varepsilon >0$ imply%
\begin{eqnarray*}
B &=&A\delta ^{\beta -\alpha } \\
\frac{1}{A} &=&\frac{\delta ^{\beta -\alpha }}{\beta }\left( z_{0}^{-\beta
}-\delta ^{-\beta }\right) +\frac{\delta ^{-\alpha }}{\alpha }.
\end{eqnarray*}

\textbf{Solution:} The smooth pasting condition implies 
\begin{equation*}
Bx^{-\beta -1}=Ax^{-\alpha -1}
\end{equation*}%
for $x=\delta $, i.e.%
\begin{eqnarray*}
B\delta ^{-\beta -1} &=&A\delta ^{-\alpha -1} \\
B &=&A\delta ^{-\alpha -1+\beta +1} \\
&=&A\delta ^{\beta -\alpha }.
\end{eqnarray*}%
The integral over the density function can be splitted,%
\begin{eqnarray*}
\int_{z_{0}}^{\infty }f_{z}(x)dx &=&\int_{z_{0}}^{\delta
}f_{z}(x)dx+\int_{\delta }^{\infty }f_{z}(x)dx \\
&=&B\int_{z_{0}}^{\delta }x^{-\beta -1}dx+A\int_{\delta }^{\infty
}x^{-\alpha -1}dx \\
&=&-\frac{B}{\beta }\delta ^{-\beta }+\frac{B}{\beta }z_{0}^{-\beta }+\frac{A%
}{\alpha }\delta ^{-\alpha }.
\end{eqnarray*}%
Substitution of $B=A\delta ^{\beta -\alpha }$ yields%
\begin{equation*}
-\frac{A\delta ^{\beta -\alpha }}{\beta }\delta ^{-\beta }+\frac{A\delta
^{\beta -\alpha }}{\beta }z_{0}^{-\beta }+\frac{A}{\alpha }\delta ^{-\alpha
}=1
\end{equation*}%
or%
\begin{equation*}
\frac{1}{A}=\frac{\delta ^{\beta -\alpha }}{\beta }z_{0}^{-\beta }-\frac{%
\delta ^{\beta -\alpha }}{\beta }\delta ^{-\beta }+\frac{\delta ^{-\alpha }}{%
\alpha }.
\end{equation*}

\item Derive the log-likelihood function of the parameters $\alpha ,\beta $
and $\delta $ (please derive it and do not simply state the log-likelihood
function, which would be too easy given that it is printed in the article).

\textbf{Solution: }In general, the loglikelihood function is%
\begin{equation*}
\ln L=\sum_{i=1}^{n}\ln f_{z}(z_{i}).
\end{equation*}%
To simplify the notation, assume that the observations are ordered such that
the first $K$ observations $z_{1},\ldots ,z_{K}$ are less than (or equal to) 
$\delta $ while the remaining $n-K$ observations are greater than $\delta $.
Then%
\begin{eqnarray*}
\ln L &=&\sum_{i=1}^{K}\ln \left( Bz_{i}^{-\beta -1}\right)
+\sum_{i=K+1}^{n}\ln \left( Az_{i}^{-\alpha -1}\right)  \\
&=&K\ln B-\left( \beta +1\right) \sum_{i=1}^{K}\ln z_{i}+\left( n-K\right)
\ln A-\left( \alpha +1\right) \sum_{i=K+1}^{n}\ln z_{i}.
\end{eqnarray*}%
Substitute $B=A\delta ^{\beta -\alpha }$, then%
\begin{eqnarray*}
\ln L &=&K\ln \left( A\delta ^{\beta -\alpha }\right) -\left( \beta
+1\right) \sum_{i=1}^{K}\ln z_{i}+\left( n-K\right) \ln A-\left( \alpha
+1\right) \sum_{i=K+1}^{n}\ln z_{i} \\
&=&K\ln A+K\left( \beta -\alpha \right) \ln \delta +\left( n-K\right) \ln A
\\
&&-\left( \beta +1\right) \sum_{i=1}^{K}\ln z_{i}-\left( \alpha +1\right)
\sum_{i=K+1}^{n}\ln z_{i} \\
&=&n\ln A+K\left( \beta -\alpha \right) \ln \delta -\left( \beta +1\right)
\sum_{i=1}^{K}\ln z_{i}-\left( \alpha +1\right) \sum_{i=K+1}^{n}\ln z_{i}.
\end{eqnarray*}

\item Load the dataset \texttt{gdpdesaster.csv}. It contains the GDP
desaster data used by Barro and Jin, i.e. 157 GDP contractions $b$ of more
than 9.5\%. Transform the desaster size according to (\ref{e1}). In R, write
a function that computes the log-likelihood as a function of the parameters $%
\alpha $ and $\beta $ for a given value of $\delta $. Set $z_{0}=1.105$ and $%
\delta =1.4$, and numerically compute the maximum likelihood estimates $\hat{%
\alpha}$ and $\hat{\beta}$ as well as the maximized value of the
loglikelihood function. Hint: Remember that the optimization command \texttt{%
optim} minimizes rather than maximizes the objective function.

\item Define a grid $1.300,1.301,1.302,\ldots ,1.600$ for the parameter $%
\delta $ and compute the maximized value of the loglikelihood function for
each grid value. Plot the log-likelihood against the grid to show that the
log-likelihood function has several local minima. Find the maximum
likelihood estimate $\hat{\delta}$.

\item Compute the asymptotic standard errors for $\hat{\alpha},\hat{\beta},%
\hat{\delta}$. Hint: Define another function that computes the
log-likelihood as a function of the parameter vector $(\alpha ,\beta ,\delta
)$.\newpage 
\end{enumerate}

\section{Estimation of observed Evans bubbles}

Evans, \textquotedblleft Pitfalls in Testing for Explosive Bubbles in Asset
Prices\textquotedblright , \emph{American Economic Review}, 81 (1991)
922-930, considers periodically collapsing bubbles. Evans bubbles are
stochastic processes described by%
\begin{equation}
B_{t}=\left\{ 
\begin{array}{ll}
\left( 1+r\right) B_{t-1}u_{t} & \text{\quad if }B_{t-1}<\alpha  \\ 
\delta +\frac{1+r}{\pi }\left( B_{t-1}-\frac{\delta }{1+r}\right) u_{t} & 
\quad \text{if }B_{t-1}\geq \alpha \text{ and }D_{t}=1 \\ 
\delta u_{t} & \quad \text{if }B_{t-1}\geq \alpha \text{ and }D_{t}=0%
\end{array}%
\right.   \label{bubble}
\end{equation}%
where $r$ is the expected growth rate, $\alpha $ is a threshold value, $%
\delta $ is the expected bubble value immediately after a crash, $\pi $ is a
probability, $u_{t}=\exp \left( y_{t}-\sigma ^{2}/2\right) $ where $%
y_{t}\sim N(0,\sigma ^{2})$ with $\sigma =0.05$, and $D_{t}$ is a Bernoulli
distributed random variable with $D_{t}=1$ (bubble continues) with
probability $\pi $ and $D_{t}=0$ (bubble bursts) with probability $1-\pi $.
Let $0<\delta <\left( 1+r\right) \alpha $.

\begin{enumerate}
\item Show that Evans bubbles are rational in the sense that%
\begin{equation*}
E(B_{t}|B_{t-1})=(1+r)B_{t-1}.
\end{equation*}%
Hint: $E(u_{t})=1$.

\textbf{Solution}: We consider the cases $B_{t-1}<\alpha $ and $B_{t-1}\geq
\alpha $ separately. The first case, $B_{t-1}<\alpha $, is straightforward,%
\begin{eqnarray*}
E\left( B_{t}|B_{t-1}\right)  &=&E\left( \left( 1+r\right)
B_{t-1}u_{t}|B_{t-1}\right)  \\
&=&\left( 1+r\right) B_{t-1}E(u_{t}) \\
&=&\left( 1+r\right) B_{t-1}.
\end{eqnarray*}%
The second case is straightforward, too,%
\begin{eqnarray*}
E\left( B_{t}|B_{t-1}\right)  &=&E\left( \left[ \delta +\frac{1+r}{\pi }%
\left( B_{t-1}-\frac{\delta }{1+r}\right) u_{t}\right] \pi +\delta
u_{t}\left( 1-\pi \right) |B_{t-1}\right)  \\
&=&E\left( \delta \pi +\left( 1+r\right) \left( B_{t-1}-\frac{\delta }{1+r}%
\right) u_{t}+\delta u_{t}\left( 1-\pi \right) |B_{t-1}\right)  \\
&=&\delta \pi +\left( 1+r\right) \left( B_{t-1}-\frac{\delta }{1+r}\right)
E\left( u_{t}\right) +\delta E\left( u_{t}\right) \left( 1-\pi \right)  \\
&=&\delta \pi +\left( 1+r\right) B_{t-1}-\frac{\left( 1+r\right) \delta }{1+r%
}+\delta \left( 1-\pi \right)  \\
&=&\left( 1+r\right) B_{t-1}.
\end{eqnarray*}%
Hence, no matter if $B_{t-1}<\alpha $ or $B_{t-1}\geq \alpha $, the
conditional expectation is $\left( 1+r\right) B_{t-1}$ in both cases. In
other words, the discounted bubble is a martingale.

\item Set $\alpha =1$, $\delta =0.5$, $\pi =0.95,\sigma =0.05$, $r=0.05$ and
simulate a path $B_{1},\ldots ,B_{5000}$ (with starting value $B_{0}=\delta $%
). Compute and plot the cumulative mean%
\begin{equation*}
\bar{X}_{i}=\frac{1}{i}\sum_{j=1}^{i}B_{i}.
\end{equation*}%
Explain why the plot indicates that the bubble has (very) heavy tails.

\item Load the dataset \texttt{evans.csv}. It contains an artificial path of
length $T=1000$ of a bubble process (of course, the bubble is only one
component of stock prices and cannot be observed separately in real
applications). Suppose the parameter $r=0.05$ and the starting value $%
B_{0}=\delta $ are known (but $\delta $ is not). In R, write a function that
computes the log-likelihood for $(\alpha ,\delta ,\pi ,\sigma )$. Hints: If $%
X\sim LN(\mu ,\sigma )$ then $aX\sim LN(\mu +\ln a,\sigma )$. In R, the
density function of the $LN(\mu ,\sigma )$ distribution at point $x$ can be
computed by \texttt{dlnorm(x,mu,sigma)}.

\textbf{Solution:} The log-likelihood function is%
\begin{equation*}
\ln L=\sum_{t=2}^{T}\ln f\left( B_{t}|B_{t-1}\right) 
\end{equation*}%
where $f(B_{t}|B_{t-1})$ is the conditional density. 

\item Now consider one-dimensional sections of the log-likelihood function.
Set $\alpha _{0}=1$, $\delta _{0}=0.5$, $\pi _{0}=0.95$, and $\sigma
_{0}=0.05$. Define a grid $g=(0.80,0.81,\ldots ,1.50)$. For each grid
element $g_{i}$ compute the log-likelihood function at $(\alpha
_{0}g_{i},\delta _{0},\pi _{0},\sigma _{0})$ and plot it against the grid.
Do the same for $(\alpha _{0},\delta _{0}g_{i},\pi _{0},\sigma _{0})$ and $%
(\alpha _{0},\delta _{0},\pi _{0},\sigma _{0}g_{i})$. As to $\pi _{0}$ use
the grid $0.900,0.901,\ldots 0.999$. Explain why standard numerical methods
to maximize the log-likelihood function are likely to fail.

\item Assume that the parameter $\alpha =0.98$ is known. Compute the maximum
likelihood estimates $\hat{\delta}$, $\hat{\pi}$, and $\hat{\sigma}$.

\item Compute the bootstrap $0.95$-confidence intervals for $\delta ,\pi ,$
and $\sigma $. Use the percentile method and set the number of bootstrap
replications to $B=1000$.\newpage 
\end{enumerate}

\section{Small sample properties of GMM estimators}

\begin{enumerate}
\item This exercise is based on the article \textquotedblleft Small-Sample
Bias in GMM Estimation of Covariance Structures\textquotedblright\ by J.G.\
Altonji and L.M.\ Segal, \emph{Journal of Business and Economic Statistics},
14 (1996) 353-366. Altonji and Segal investigate the finite-sample
properties of GMM estimators. They estimate a highly simplified artificial
example. The connection to GMM estimation is not obviously visible, but
exists.
\end{enumerate}

Let $Y_{it}$, $i=1,\ldots ,N$, $t=1,\ldots ,T$, be a panel sample. The
number of individuals is $N=10$, the number of periods is $T=50$. The sample
elements $Y_{it}$ are i.i.d. (both over individuals and time) with 
\begin{equation}
Y_{it}=\frac{X_{it}-1.648721}{2.161197}  \label{lnstd}
\end{equation}%
where $X_{it}$ are i.i.d. lognormally distributed with parameters $\mu =0$
and $\sigma =1$. Location and scale in (\ref{lnstd}) are chosen such that $%
E(Y_{it})=0$ and $Var(Y_{it})=1$.\footnote{%
The matrix of observations is generated by \texttt{y \TEXTsymbol{<}-
matrix((rlnorm(N*TT)-1.648721)/2.161197,N,TT)} where \texttt{N=10} and 
\texttt{TT=50}. Row $i$ contains the observations of individual $i$.}

The unknown parameter to be estimated is the variance $\theta $ of $Y_{it}$
(the true value of which is, of course, 1). Altonji and Segal compare two
GMM estimators of $\theta $. Both are based on the $N$ estimators of the $N$
individual variances $\sigma _{i}^{2}$, 
\begin{equation*}
\hat{\sigma}_{i}^{2}=\frac{1}{T-1}\sum_{t=1}^{T}\left( Y_{it}-\bar{Y}%
_{i}\right) ^{2}.
\end{equation*}%
The $N$ variance estimators are collected in a vector $\hat{\sigma}^{2}$.
Define a vector $\iota =(1,\ldots ,1)^{\prime }$ of length $N$. The first
GMM estimator $\hat{\theta}_{1}$ of the total variance $\theta $ is simply
the unweighted mean of the $N$ individual variance estimators (the GMM
weighting matrix is the identity matrix),%
\begin{equation*}
\hat{\theta}_{1}=\frac{1}{N}\sum_{i=1}^{N}\hat{\sigma}_{i}^{2}=\left( \iota
^{\prime }\iota \right) ^{-1}\iota ^{\prime }\hat{\sigma}^{2}.
\end{equation*}%
The second GMM estimator $\hat{\theta}_{2}$ is based on the estimated
asymptotically optimal weight matrix,%
\begin{equation*}
\hat{\theta}_{2}=\left( \iota ^{\prime }\hat{\Omega}^{-1}\iota \right)
^{-1}\iota ^{\prime }\hat{\Omega}^{-1}\hat{\sigma}^{2}.
\end{equation*}%
The estimated weight matrix $\hat{\Omega}$ contains the variances and
covariances of the individual variance estimators $\hat{\sigma}_{i}^{2}$.
The element $(i,j)$ of $\hat{\Omega}$ is (Altonji and Segal use a slightly
different normalization)%
\begin{equation*}
\hat{\Omega}_{ij}=\widehat{Cov}(\hat{\sigma}_{i}^{2},\hat{\sigma}_{j}^{2})=%
\frac{\left[ \frac{1}{T-1}\sum_{t}\left( Y_{it}-\bar{Y}_{i}\right)
^{2}\left( Y_{jt}-\bar{Y}_{j}\right) ^{2}\right] -\hat{\sigma}_{i}^{2}\hat{%
\sigma}_{j}^{2}}{T}.
\end{equation*}

\begin{enumerate}
\item Write an R program to estimate the finite-sample distribution of $\hat{%
\theta}_{1}$ and $\hat{\theta}_{2}$ by simulations and plot both
distribution functions.

\item Compute the mean squared error of $\hat{\theta}_{1}$ and $\hat{\theta}%
_{2}$ (i.e. the mean squared deviation from the true value 1).

\item Load the dataset \texttt{smallgmm.csv} into the dataframe \texttt{x}
and convert it into a matrix using the command \texttt{y \TEXTsymbol{<}-
matrix(x[,1],10,50)}. Compute $\hat{\theta}_{2}$ for this dataset. Use the
nonparametric bootstrap\footnote{%
Resampling can be done by \texttt{ystar \TEXTsymbol{<}-
matrix(sample(y,size=500,replace=TRUE),10,50)}.} to estimate the bias and
standard error of $\hat{\theta}_{2}$.\newpage 
\end{enumerate}

\section{Estimation of the Cox-Ingersoll-Ross model}

Cox, Ingersoll and Ross, \textquotedblleft A Theory of the Term Structure of
Interest Rates\textquotedblright , \emph{Econometrica} 53 (1985) 385-407,
suggest a continuous-time model for the short-term interest rate. The
stochastic process is described by the stochastic differential equation%
\begin{equation}
dX_{t}=\left( \theta _{1}-\theta _{2}X_{t}\right) dt+\theta _{3}\sqrt{X_{t}}%
dW_{t}  \label{sdecir}
\end{equation}%
where $W_{t}$ is a standard Wiener process and $X_{0}>0$.

\begin{enumerate}
\item Activate the R package \texttt{sde}. Generate and plot a single path
on the time interval $[0,200]$ of an Cox-Ingersoll-Ross process with
parameters $\theta _{1}=0.03$, $\theta _{2}=0.5$, and $\theta _{3}=0.08$ and
starting value $X_{0}=0.06$ using the command \texttt{sde.sim}. Set the
number of steps to $N=200$.

\item Continuous-time models are sometimes estimated by discretizing them in
a crude way. The discretized version of (\ref{sdecir}) is, of course,%
\begin{equation}
X_{t}=X_{t-1}+\left( \theta _{1}-\theta _{2}X_{t-1}\right) +\theta _{3}\sqrt{%
X_{t-1}}\varepsilon _{t}  \label{dissde}
\end{equation}%
with $\varepsilon _{t}\sim N(0,1)$ and starting value $X_{0}=\theta
_{1}/\theta _{2}$. Find estimators for the parameters $\theta _{1},\theta
_{2},\theta _{3}$ in (\ref{dissde}) that can be computed fast (e.g. least
squares estimators).

\item Load the dataset \texttt{cirpath.csv}. The process is not observed
continuously. The dataset only contains observations of $X_{t}$ at discrete
time points $t=1,\ldots ,200$. Estimate the parameters $\theta _{1},\theta
_{2}$, and $\theta _{3}$ by indirect inference with the auxiliary model (\ref%
{dissde}). Assume that the (unobserved) starting value is $X_{0}=\theta
_{1}/\theta _{2}$.
\end{enumerate}

\end{document}
